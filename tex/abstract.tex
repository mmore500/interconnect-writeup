\begin{abstract}

Constructing and studying artificial evolutionary systems that aim to indefinitely produce artifacts of continuing novelty and increasing complexity has proven --- and doubtlessly will continue on as  --- a rich vein for practical, scientific, philosophical, and artistic innovations.
Recent evidence has suggested existing computational artificial life systems might be meaningfully constrained by practical limitations on simulation scale.
Ackley's concept of indefinite scalability describes constraints on open-ended systems neccessary to incoroporate theoretically unbounded computational resources.
Here, we argue that as a bridge to true indefinite scalability, practical scalability should also be considered: how to design open-ended evolutionary systems to make effective use of existing, commercially-available parallel- and distributed-computing hardware.
%We review existing in , and open-ended systems should emphasize interaction between parallel hardware components.
We highlight log-time hardware interconnects as a potentialy fruitful tool for  practical scalability.
We prove several results about scaling relationships of per-component traffic load of small-world interaction networks embedded on computational mesh.
In some, but not all cases, log-time physical interconnects yield better asymptotic scaling behavior.
Then, we turn our attention to how digital evolution systems might be constructed to exploit of physical log-time interconnects.
We describe an extension to the DISHTINY digital multicellularity framework that  allows cells to establish long-distance cell-cell interconnects that, in implementation, could take advantage of log-time physical interconnects.
We examine two case studies of evolved strains, showing how adaptively exploit these interconnects.
\end{abstract}
