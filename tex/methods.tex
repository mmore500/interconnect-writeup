\section{Methods}

Our evolutionary case studies employ an extension to the DISHTINY framework for studying fraternal transitions in individuality.
Initial work with this system characterized selective pressures for cooperation with kin \citep{moreno2019toward}.
Upcoming work extends the system to use the SignalGP event-driven genetic programming techniaue \citep{lalejini2018evolving} to control cell behaviors.
Diverse multicellular life histories evolved in SignalGP-enabled DISHTINY evolutionary trials, involving reproductive division of labor, resource sharing (including, in some treatments, endowment of offspring  groups), asymmetrical within-group and inter-group phenomena mediated by cell-cell messaging, morphological patterning, gene-regulation mediated life cycles, and adaptive apoptosis \citep{TODOcitedishtinygp}.

DISHTINY simulates individual cells, each of which occupies a tile on a toroidal grid.
Cells can reproduce, placing daughter cells into adjoining tiles.
We allow cells the opportunity to engage with kin in a cooperative resource-collection task (Supplementary Section \ref{sec:resourcecollection}), which can increase their individual cellular reproduction rates.
Kin groups are explicitly registered: on birth, a cell is either added to its parents group or expelled to establish a new group (Supplementary Section \ref{sec:hierarchicalnesting}).

Cells can differentiate between neighbors that are members of their kin group and neighbors that are not and alter their behavior accordingly.
Each cell consists of four SignalGP instances (all executing the same genetic program), one of which controls cell behavior with respect to each neighbor.
These instances may communicate with one another by means of intracellular messaging.
In this work, we add a fifth SignalGP instance to the DISHTINY cell.
This instance can execute special instructions to establish long-distance interconnects with other cells and engage in resource-sharing and/or message passing with those cells.
Figure \ref{fig:spiker_pointer_hardware} summarizes how SignalGP hardware is arranged within DISHTINY cells.

Long-distance interconnects are established through a developmental process, summarized in Figure \ref{fig:spiker_diagram}.
The process begins with the placement of two independent search prongs at the originating cell \ref{fig:spiker-generate}.
Each prong performs a random walk over the originating cell's kin group, accumulating positive or negative feedback based on tags expressed by underfoot cells \ref{fig:spiker-walk}.
If a prongs accumulates positive feedback too slowly, it is reset to the location of the better-scoring prong \ref{fig:spiker-reset}.
Once a positive feedback threshold has been reached, the best-scoring prong develops into a full-fledged connection \ref{fig:spiker-connect}.
At this point, the originating cell can begin exchanging messages and/or resource over the connection \ref{fig:spiker-transmit}.
Established interconnects may be subsequently removed by either participating cell \ref{fig:spiker-remove}.

Full details on hardware-level instructions and event-driven environmental cues available to cells are provided in Supplementary Sections \ref{sec:neighborinstlib}, \ref{sec:neighboreventlib}, \ref{sec:interconnectinstlib}, and \ref{sec:interconnecteventlib}.

\begin{figure}[t]
\begin{center}
\includegraphics[width=0.7\linewidth]{spiker-pointer-hardware}
\caption{
Arrangement of SignalGP hardware within DISHTINY cells (gray squares).
Neighbor-managing hardware (circles) receive stimuli and control cell behavior with respect to a particular cell neighbor.
Network-managing hardware (interior squares) receive stimuli and controll cell behavior with respect to more distant neighbors a cell has established interconnects with.
}
\label{fig:spiker_pointer_hardware}
\end{center}
\end{figure}


% OLD VERSION IS COMMENTED OUT; NEW VERSION BELOW!

\iffalse

\begin{figure}[!htbp]
\begin{center}

\begin{minipage}{\linewidth}
\begin{center}
\begin{subfigure}[t]{0.25\linewidth}
  \includegraphics[width=\linewidth,trim={0 100 100 0},clip]{spiker-diagram/spiker-generate}
  \caption{cell buds developmental search prongs}
  \label{fig:spiker-generate}
\end{subfigure}
	\hspace{2ex}
\begin{subfigure}[t]{0.25\linewidth}
  \includegraphics[width=\linewidth,trim={0 100 100 0},clip]{spiker-diagram/spiker-walk}
  \caption{search prongs perform random walk}
  \label{fig:spiker-walk}
\end{subfigure}
  \hspace{2ex}
\begin{subfigure}[t]{0.25\linewidth}
  \includegraphics[width=\linewidth,trim={0 100 100 0},clip]{spiker-diagram/spiker-swap}
  \caption{poorly-performing prong resets to better-performing prong}
  \label{fig:spiker-swap}
\end{subfigure}
\end{center}
\end{minipage}

\begin{minipage}{\linewidth}
\begin{center}
\begin{subfigure}[t]{0.25\linewidth}
  \includegraphics[width=\linewidth,trim={0 100 100 0},clip]{spiker-diagram/spiker-connect}
  \caption{search prong matures into established connection}
  \label{fig:spiker-connect}
\end{subfigure}
  \hspace{2ex}
\begin{subfigure}[t]{0.25\linewidth}
  \includegraphics[width=\linewidth,trim={0 100 100 0},clip]{spiker-diagram/spiker-transmit}
  \caption{cells exchange messages over established interconnect}
  \label{fig:spiker-transmit}
\end{subfigure}
   \hspace{2ex}
\begin{subfigure}[t]{0.25\linewidth}
  \includegraphics[width=\linewidth,trim={0 100 100 0},clip]{spiker-diagram/spiker-remove}
  \caption{established interconnects may be removed by either participating cell}
  \label{fig:spiker-remove}
\end{subfigure}
\end{center}
\end{minipage}

\caption{
Illustration of the developmental process used to establish long-distance interconnects.
You can see this developmental process in action in an evolved strain at \url{https://mmore500.com/hopto/ap}.
}
\label{fig:spiker_diagram}
\end{center}
\end{figure}

\fi




%%% HERE IS THE REAL FIGURE!
%\iffalse

\begin{figure}[!htbp]
\begin{center}
\includegraphics[width=1.0\linewidth]{img/spiker-diagram/spiker-combined.pdf}
\caption{
Illustration of the developmental process used to establish long-distance interconnects.  Cells start by budding developmental search prongs (a) that perform a random search (b), reverting to the most successful search (c) where it matures to establish a connection (d).  Messages and resources can be transmitted over a connection (e) until either cell decides to terminate the connection (f).
You can see this developmental process in action in an evolved strain at \url{https://mmore500.com/hopto/ap}.
}
\label{fig:spiker_diagram}
\end{center}
\end{figure}

%\fi


\subsection{Evolutionary Screens}

Our evolutionary screens consisted of 64 independent evolutionary batches.
Each batch was processed in four-hour epoch to enable efficient job scheduling.
Each batch consisted of four isolated 45-by-45 toroidal subpopulations.
Subpopulations were completely intermixed in between four-hour steps.
To facilitate evolutionary search, in addition to a base mutation rate applied to cell division, additional mutation was applied to cells seeding a toroidal grid at the outset of an epoch or budding to form new kin groups during an epoch.

We screened across four-hour checkpoints of replicate batches to see if messages or resource were being sent over interconnects.
We sampled from these populations, performing screens for knockouts of over-interconnect messaging or resource sharing.
Strains with adaptive over-interconnect messaging or resource sharing were then secondarily screened to determine if re-routing messages or shared resource decreased fitness.

Fitness was measured relatively using competition experiments between strains.
For many competition experiments reported in the case studies, we provide hyperlinks to load a in-browser DISHTINY simulation with the actual strains that were used.
In this web viewer, wild-type strains carry phylogentic root ID 1 and knockout strains carry ID 2.

\subsection{Implementation}

We implemented our experimental system using the Empirical library for scientific software development in C++, available at \url{https://github.com/devosoft/Empirical} \citep{charles_ofria_2019_2575607}.
We used OpenMP to parallelize our main evolutionary replicates, distributing work over two threads.
The code used to perform and analyze our experiments, our figures, data from our experiments, and a live in-browser demo of our system is available via the Open Science Framework at \url{https://osf.io/53vgh/} \citep{foster2017open}.
