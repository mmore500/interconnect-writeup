\section{Cell-Level Organisms in DISHTINY} \label{sec:cell_level_organisms}

SignalGP programs are collections of independent procedural functions, each equipped with a bit-string tag \cite{lalejini2018evolving}.
A function is triggered by a signal with affinity that maximally and sufficiently matches its tag.
(A binding threshold of 0.1 was used in these experiments).
Signals may be generated by the environment, received as messages from other agents, or triggered internally by function execution.
Signals, and the ensuing chains of procedural execution they give rise to, are processed pseudo-concurrently by 24 virtual CPU cores.
Figure \ref{fig:signalgp-cartoon} schematically depicts a single SignalGP instance.

In this work, we introduce a regulatory extension to the SignalGP system.
During runtime, instructions may increase or decrease each tagged function's intrinsic tendency to match with --- and activate in response to --- tagged queries.
Intrinsic tag-to-tag match distances $m$ are modulated by a regulator value $r$ (baseline, 1.0) to become $r + r \times m$.
This scheme allows a function to be upregulated such that every query activates that function (e.g., $r = 0$) or no query activates that function (e.g., $r = \texttt{inf}$).
These regulation settings are heritable during reproduction but automatically decay after a number of updates determined when they are set.

To allow cells to protect themselves form potentially antagonistic interactions with their neighbors, we filter intercellular messages through a tag-matching membrane.
At runtime, cells can embed tags in this membrane that either admit or repel incoming messages.
Messages that do not match with a membrane tag are repelled.
A message, for example, that would activate a SignalGP function containing an apoptosis instruction could be rejected while other messages are accepted.
Tags embedded in this membrane automatically decay and may also be regulated.
We also filter messages between hardware instances within the same cell through a tag-matching membrane, but the default behavior for messages with unmatched tags is admission rather than rejection.

Previous work evolving digital organisms in grid-based problem domains has relied on a single computational instance which designates a direction to act in via an explicit cardinal ``facing'' state or output \cite{goldsby2014evolutionary, goldsby2018serendipitous, grabowski2010early, biswas2014causes, lalejini2018evolving}.
Under this paradigm, a large portion of genotype space encodes behaviors that are intrinsically asymmetrical with respect to absolute or relative (depending on implementation) cardinal direction.
However, in grid-based tasks, directional phenotypic symmetry is generally advantageous.
That is --- in the absence of a polarizing external stimulus --- successful agents generally behave uniformly with respect to each cardinal direction of the grid.
In this work, each cell employs four instances of SignalGP hardware: one ``facing'' each cardinal direction.
These computational instances all execute the same SignalGP program but are otherwise decoupled and may follow independent chains of execution and develop independent regulatory states.
Instances within a cell execute round robin step-by-step in an order that is randomly drawn at the outset of each update.

Genetic encodings that exploit problem-domain symmetries are known to promote evolvability and --- ultimately --- evolved solution quality \cite{clune2011performance, cheney2014unshackling}.
We submit that this directional hardware replication protocol likely increases the fraction of genotype space that encodes cardinally-symmetric phenotypes and therefore better facilitates the evolution of high-fitness phenotypes.
In further work, we look forward to exploring the evolvability and solution quality implications of this new approach.

The single SignalGP program that is mirrored across the cell's computational instances represents the cell's genome.
Mutation, with standard SignalGP mutation parameters as in \cite{lalejini2018evolving}, is applied to 1\% of daughter cells at birth.
In addition, genomes encode the bitstrings associated with environmental events.
These bitstrings evolve at a per-bit mutation rate equivalent to the bitstring labels of SignalGP functions.

Instances within a cell may send intracellular messages to one another or intercellular messages to a neighboring cell.
Intercellular messages are received by the SignalGP instance that faces the sending cell.
Figure \ref{fig:dishtinygp-cartoon} schematically depicts the configuration of the four SignalGP instances that constitute a single DISHTINY cell as well as the instances of neighboring cells that receive extracellular messages from the focal cell.


