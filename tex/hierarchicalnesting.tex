\section{Hierarchical Nesting} \label{sec:hierarchical_nesting}

Hierarchical levels are introduced into the system through multiple separate, but overlaid, instantiations of this resource wave/channel-signaling scheme.
We refer to each independent resource wave/channel-signaling system as a ``level.''
In some experimental treatments, we allowed two resource wave/channel-signaling levels, identified here as level one and level two.
On level one, resource waves extended a radius of two toroidal tiles.
On level two they extended a radius of six toroidal tiles.
On both levels, activated cells netted $+0.2$ resource from a resource wave, but did not collect any resource outside the extent of the resource wave.
Due to the different radii of resource waves on different levels, level one selects for small same-channel signaling networks and level two selects for large same-channel signaling networks.

Each cell contained a pair of separate channel IDs, the first for level one and the second for level two.
We kept these channel IDs hierarchically nested by constraining inheritance during reproduction.
Daughter cells could not inherit just the level-one channel ID, they could either
\begin{enumerate}
\item inherit both level-one and level-two channel ID,
\item inherit level-two channel ID but not level-one channel ID, or
\item inherit neither channel ID.
\end{enumerate}
Hierarchically nested channel IDs are analogous to a strict corporate organizational structure: all employees (i.e., cells) are members of one department (i.e., level-one channel network) and one corporation (i.e., level-two channel network) but no employee can be a member of two departments and no department can be a member of two corporations.
Figure \ref{fig:morphology-wt} depicts hierarchically nested channel states assumed by an evolved strain.

An evolutionary transition in individuality can readily be evaluated within the DISHTINY framework with respect to same-channel network groups.
In addition to a potentially functionally cooperative relationship, shared channel IDs --- which may only systematically arise through inheritance --- imply a close hereditary relationship.
Because new channel IDs arise first in a single cell, same-channel signaling networks are reproductively bottlenecked analogously to a "Staying Together" life cycle (rather than a "Coming Together" life cycle) \citeinappendix{staps2019emergence}.
This precludes chimeric groups, except for mutations arising from somatic reproduction and rare cases of channel ID collision.

To recognize an evolutionary transition in individuality, we can evaluate
\begin{enumerate}
\item whether cells with the same channel ID cooperate altruistically by assessing, for example, resource sharing, and
\item whether division of reproductive labor arises by assessing whether interior cells cede reproduction to those at the periphery.
\end{enumerate}
If cells sharing the same level-one channel fulfill these conditions, we would suppose that a first-level transition in individuality had occurred.
Likewise, if cells sharing the sharing the same level-two channel fulfill these conditions, we would suppose that a second-level transition in individuality had occurred.
Further, we can screen for the evolution of complex multicellularity by assessing cell-cell messaging, regulatory patterning, and functional differentiation between cells within the a same-channel signaling network \citeinappendix{knoll2011multiple}.


