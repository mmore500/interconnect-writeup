\subsection{Wiring an Ideal Space-Filling Hierarchical Tree without Log-Time Physical Interconnects}

Consider, again, a set of computational nodes arranged in a $r$-dimensional mesh.
In each dimension, physical interconnects run between immediately adjacent pairs of nodes.
Let this physical hardware corresponds to a graph $N$ where $V(N)$ represents computational nodes and $E(N)$ represents physical interconnects between nodes.

Let $d(a,b)$ represent the typical number of physical interconnects traversed on a shortest path between a pair of arbitrary nodes $a, b \in V(N)$.
This is conceptually equivalent to Manhattan distance.

Suppose we have a small-world graph $G$ with maximum vertex degree bounded by a finite constant $m$.
This vertices of this graph $G$ are embedded one-to-one on $N$ such that $|V(G)| = |V(N)|$.

Pick an arbitrary vertex $a \in V(G)$.
By the definition of a small-world graph,
\begin{align*}
  \frac{1}{|V(G)|} \sum_{v \in V(G)} d(a, v) \propto \log |V(G)|.
\end{align*}

Because the degree of the graph $G$ is bounded by $m$, there must be a subset $T \subseteq G$ that, for some $k \geq 2$ forms a complete $k$-nary tree rooted at $a$ such that
\begin{enumerate}
  \item the tree height of $T$ is $h \propto \log |V(G)|$ and
  \item $|V(T)| \propto |V(G)|$.
\end{enumerate}

Legenstein and Maass \citep{legenstein2001optimizing} establish a lower bound for the length of wiring required to construct $k$-nary tree with $n$ nodes on a a one-dimensional $L_1$ grid,
\begin{align*}
\Omega(n \log n).
\end{align*}
In our case, this corresponds to the total number of hops over $N$ to traverse every edge in $T$.
Because, $T \subseteq G$, $\Omega(n \log n)$ is also a best-case lower bound for the total number of hops over $N$ to traverse every edge in $G$.

Because the degree of vertices in $V(T)$ is bounded by $k$,
\begin{align*}
|E(T)| \in O \Big( |V(T)| \Big).
\end{align*}.

In fact, because the degree of vertices in $V(G)$ is also bounded by $m$,
\begin{align*}
|E(G)| \in O \Big( |V(G)| \Big).
\end{align*}.

Let the wiring cost of an edge $\{x, y\}$ in $E(G)$ refer to the number of hops over $N$ required to travel from $x$ to $y$.
The best-case average wiring cost per edge can be computed as the best-case total wiring cost divided by the worst-case number of edges.
For arbitrary $\{x, y\} \in E(G)$,
\begin{align*}
\bar{d}(x, y)
&\in \Omega \Big( \frac{ |E(G)| \times \log |E(G)| }{ |E(G)| } \Big)\\
&\in \Omega \Big( \log |E(G)| \Big).
\end{align*}

This result applies to all possible small-world graphs $G$ embedded on a one-dimensional computational mesh.

To tractably extend our analysis to three-dimensional meshes, rather than all small-world graphs we will specifically analyze the wiring cost of ideal space-filling trees \citep{kuffner2009space}.
This construction efficiently distributes elements of $G$ over $N$ with respect to wiring cost.
This construction potentially represents a lower bound on wiring cost, its optimality has not been concretely established.

For three dimensions, the total length of wiring required as a function of the number of nodes is
\begin{align*}
w_3(n)
&=
\sum_{i=1}^{\log_8 n} \Big[
  \frac{n}{8^i} % how many to draw
  \times
  \frac{3}{2} \times 8 \times 2^{i} % how big each one is
\Big].
\end{align*}

Because
\begin{align*}
\lim_{n \rightarrow infty}
\frac{w_3(n)}{n} = 4,
\end{align*}

we have $w_3(n) \in \Theta \Big( n \Big)$.
For a $n$-node tree, edge count also $|E(G)| \in \Theta \Big( |V(G)| \Big)$.
So, average edge wiring cost remains constant as $|V(G)|$ scales.
Similar analysis concludes an equivalent result in the two-dimensional case.

% For two dimensions, the total wiring cost of $G$ as a function of the number of nodes is
% \begin{align*}
% w_2(n)
% &=
% \sum_{i=1}^{\log_4 n} \Big[
%   \frac{n}{4^i} % how many to draw
%   \times
%   2 \times 2^{i} % how big each one is
% \Big].
% \end{align*}


% Evaluating the limits,
% \begin{align*}
% \lim_{n \rightarrow infty}
% \frac{w_2(n)}{n} = 2
% \end{align*}

% and

% So, the average edge length is at least
% \begin{align*}
% \frac{n}{n} = c
% \end{align*}
