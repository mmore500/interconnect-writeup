\subsection{Node-to-Node Hops per Small World Graph Connection, Hierarchical Construction}

Suppose we have a small-world graph $G$ with degree bounded by a finite constant $m$.

Pick an arbitrary vertex $a \in V(G)$.
Let $f(a)$ represent the fraction of nodes in the graph $V(G)$ that
By the definition of a small-world graph,
\begin{align*}
  \frac{1}{|V(G)|} \sum_{v \in V(G)} d(a, v) \propto \log |V(G)|.
\end{align*}

Because the degree of the graph $G$ is bounded by $m$, there must be a subset $T \subseteq G$ that, for some $k \geq 2$ forms a complete $k$-nary tree rooted at $a$.
Specifically, such that
\begin{enumerate}
  \item $E(T) \subseteq E(G)$,
  \item $V(T) \subseteq V(G)$,
  \item for $T$, tree height $h \propto \log |V(G)|$,
  \item $|V(T)| \propto |V(G)|$.
\end{enumerate}

Legenstein and Maass establish a lower bound for the amount of wiring required to construct a one-dimensional $k$-nary tree with $n$ nodes: $\Omega(n \log n)$ \citep{legenstein2001optimizing}.
This corresponds to a best-case total edge length of $T$.
Because, $T \subseteq G$, $\Omega(n \log n)$ is also a best-case lower bound for the total edge length of $G$.

Because the degree of nodes in $T$ is bounded by $k$,
\begin{align*}
|E(T)| \in O \Big( |V(T)| \Big).
\end{align*}.

In fact, because the degree of nodes in $G$ is also bounded by $m$,
\begin{align*}
|E(G)| \in O \Big( |V(G)| \Big).
\end{align*}.

The best-case average edge length can be computed as the best-case total edge length divided by the worst-case number of edges.
For arbitrary $\{x, y\} \in E(G)$,
\begin{align*}
\bar{d}(x, y) &
\in \Omega \Big( \frac{ |E(G)| \times \log |E(G)| }{ |E(G)| } \Big)\\
\in \Omega \Big( \log |E(G)| \Big).
\end{align*}

This result applies to all possible small-world graphs in a one-dimensional case.

To analyze two-dimensional cases, we will analyze the wiring cost of regular space-filling trees with construction based on \citep{kuffner2009space}.
This represents an efficient , potentially representative of a lower bound, but its optimality has not been concretely established.

For two dimensions, the total length of wiring required as a function of the number of nodes is
\begin{align*}
w_2(n)
&=
\sum_{i=1}^{\log_4 n} \Big[
  \frac{n}{4^i} % how many to draw
  \times
  2 \times 2^{i} % how big each one is
\Big].
\end{align*}

For three dimensions, the total length of wiring required as a function of the number of nodes is
\begin{align*}
w_3(n)
&=
\sum_{i=1}^{\log_8 n} \Big[
  \frac{n}{8^i} % how many to draw
  \times
  \frac{3}{2} \times 8 \times 2^{i} % how big each one is
\Big].
\end{align*}

Evaluating the limits,
\begin{align*}
\lim_{n \rightarrow infty}
\frac{w_2(n)}{n} = 2
\end{align*}

and
\begin{align*}
\lim_{n \rightarrow infty}
\frac{w_3(n)}{n} = 4
\end{align*}

So, the total length of wiring in both cases
\begin{align*}
L \in \Omega \Big( n )
\end{align*}

For a $n$-node tree, the number of edges is given as
\begin{align*}
\sum_{l = 1}^{\log n} b^{l}
&= b^{\log n + 1} - 1 \\
&= b \times b^{\log n} \\
&= b \times n - 1
\end{align*}

So, the average edge length is at least
\begin{align*}
\frac{n}{n} = c
\end{align*}
