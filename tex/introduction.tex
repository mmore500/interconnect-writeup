\section{Introduction}

The challenge, and promise, of open-ended evolution has animated decades of inquiry and discussion within the artificial life community \citep{packard2019overview}.
The difficulty of devising models that produce characteristic outcomes of open-ended evolution suggests profound philosophical or scientific blind spots in our understanding of the natural processes that gave rise to contemporary organisms and ecosystems.
Already, pursuit of open-ended evolution has yielded paradigm-shifting insights.
For example, novelty search, which demonstrated how processes promoting non-adaptive diversification can ultimately yield adaptive outcomes that were previously unattainable \citep{lehman2011abandoning}.
Such work lends insight to fundamental questions in evolutionary biology, such as the role --- or agnosticism -- of natural selection with respect to increases in complexity \citep{lehman2012evolution, lynch2007frailty} and the origins of evolvability \citep{lehman2013evolvability, kirschner1998evolvability}.
Evolutionary algorithms devised in support of open-ended evolution models also promise to deliver tangible broader impacts on society.
Possibilities include the generative design of engineering solutions, consumer products, art, video games, and AI systems \citep{nguyen2015innovation, stanley2017open}.

Preceding decades have witnessed advances toward defining --- quantitatively and philosophically --- the concept of open-ended evolution \citep{lehman2012beyond, dolson2019modes, bedau1998classification}, as well as investigating causal phenomena that promote it such as ecological dynamics, selection, and evolvability \citep{dolson2019constructive, soros2014identifying, huizinga2018emergence}.
Together, methodological and theoretical advances have begun to yield evidence that the generative potential of artificial life systems is --- at least in part --- meaningfully constrained by available compute resources \citep{channon2019maximum}. % talk to Alex about this`

\subsection{Advances in Modern Compute Resources Rely on Distribution and Parallelism}

Since approximately the turn of the century, advances in the clock speed of traditional serial processing have trailed off \citep{sutter2005free}.
Since that time, have existing technologies begun to encounter fundamental constraints including power use and thermal dissipation \citep{markov2014limits}.
Instead, hardware innovation began to revolve around multiprocessing  \citep[p.55]{hennessy2011computer} and hardware acceleration (e.g., GPU; TODO cite).
For scientific and engineering applications, individual multiprocessors and accelerators are joined together with fast interconnects to yield so-called high performance computing clusters \citep[p.436]{hennessy2011computer}.
Scaling up artificial life systems designed to study open-ended evolution will require taking advantage of modern parallel and distributed compute technologies.

\subsection{Leveraging Parallel and Distributed Hardware in Digital Evolution}

In fact, digital evolution practitioners have a rich history of leveraging parallel and distributed hardware.
It is common practice to distribute multiple self-isolated instantiations of evolutionary runs over multiple hardware units.
In scientific contexts, this practice yields replicate datasets that provide statistical power to answer a research question \citep{dolson2017spatial}.
In applied contexts, this practice yields many converged populations that can be scavenged for the very best solutions \citep{hornby2006automated}.

Another established practice is to use ``island models'' where individuals are transplanted between populations that are otherwise independently evolving across distributed hardware.
Koza and collaborators' genetic programming work with a 1,000-cpu Beowulf cluster typifies this approach \citep{bennett1999building}.

In recent years, Sentient Technologies spearheaded digital evolution projects on an unprecedented computational scale, comprising over a million CPUs and capable of a peak performance of 9 petaflops \citep{miikkulainen2019evolving}.
According to its proponents, the scale and scalability of this DarkCycle system was a key aspect of its conceptualization \citep{gilbert_2015}.
Much of the assembled infrastructure was pieced together from heterogeneous providers and employed on a time-available basis \citep{blondeau2012distributed}.
Unlike the island model where selection events are performed independently on each CPU, this scheme apparently involved the dynamic transfer of evaluation criteria between computational instances (in addition to individual genomes) \citep{hodjat2013distributed}.

Also notable from Sentient Technologies was large-scale use of many massively-parallel hardware accelerators (e.g., 100 GPUs) to evaluate the performance of candidate deep learning neural network architectures on image classification, language modeling, and image captioning problems.
Hardware parallelism accelerated the deep learning training process used to evaluate individual solutions \citep{miikkulainen2019evolving}.
Analogous work parallelizing the evaluation of an evolutionary individual over multiple test cases in the context of genetic programming has used GPU hardware and vectorized CPU operations \citep{harding2007fast2, langdon2019continuous}.
 %TODO check this sentence

% \citep{langdon2010large} breast cancer

Existing applications of parallel and distributed computing to digital evolution work by distributing and concurrently processing populations, evolutionary individuals, or even single evolutionary individuals.
Simulation elements spread over distributed hardware components interact minimally.
Task independence, which facilitates simple, efficient implementation, is typically a key reason researchers choose to target a system for parallelization in the first place. %TODO cite?
However, distributed simulation elements are not necessarily entirely independent.
Island models transfer genomes (and sometimes evaluation criteria) between populations.
Parallelizing evaluation of a single evolutionary individual often emphasizes data-parallelism over test or training cases, which must be consolidated into a single fitness profile or deep learning weight update. %TODO tryna include lexicase
With respect to model parallelism, Harding has notably applied GPU acceleration to cellular automata models of artificial development systems, which involve intensive interaction between spatially-distributed instantiation of a genetic program  \citep{harding2007fast}.
However, in systems where evolutionary individuals themselves are parallelized they are typically completely isolated from each other.

\subsection{Leveraging Parallel and Distributed Hardware for Open-Ended Evolution}

%TODO the system should be capable of distributing individuals?
% individuals evolving in environments where they have to deal with the issues of being distributed (but not necessarily distributed on hardware although they could be)
% * implementation of system for efficiency
% * agents creating algorithms to deal with being distributedy
% "leaves the door open"

% rephrase tongue twister
% add citations
% generative developmental processes (very broad, think about)
% reading from one genome in multiple instances allows for repetitions, symmetries, hierarchy (cite Clune?)

Unlike most existing applications of parallel and distributed computing to digital evolution, open-ended evolution researchers should prioritize dynamic interactions between distributed simulation elements.

Parallel and distributed computing enables larger populations and metapopulations.
However, ecologies, co-evolutionary dynamics, social behavior, and culture necessitate dynamic interactions between contemporary evolving individuals.

Parallel and distributed computing might also enable more computationally intensive (potentially, but not necessarily, more complex) evolutionary individuals.
However, developmental processes and emergent functionality necessitate dynamic interactions between components of an evolving individual. %emergent consolidated functionality?
Even at a scale where evolutionary individuals remain computationally tractable on a single hardware element, modeling individuals as a collection of distributed components may promote germane properties such as modularity (TODO cite), exploratory processes \citep{gerhart2007theory}, phenotypic symmetries \citep{stanley2003taxonomy}. %TODO MORE things here??
(This approach also leaves the door open to larger-scale individuals in the future.)

David Ackley has led toward such a distributed approach, laying out an ambitious vision for modular hardware at a theoretically unlimited scale \citep{ackley2011pursue} and demonstrating an algorithmic substrate for emergent agents that can take advantage of it \citep{ackley2018digital}.
% text from blog post!

\subsection{A Path of Expanding Computational Scale}

% potential, pragmatic?
% ROAD TO BLAH
% STEPPING STONES TO BLAHs
% JOURNEY
% INTERMEDIATE STEPS
% path
% route
% steps
% meantime
% interim

%(TODO I've heard similar discussion made before somewhere, cite it)
While by no means a certainty, the idea is that orders-of-magnitude differences in compute power will open up qualitatively different possibilities with respect to open-ended evolution is also not entirely unfounded.
Analogy to spectacular advances achieved with artificial neural networks over the last decade illuminates a possible path toward this outcome.
As with digital evolution, artificial neural networks (ANNs) were traditionally understood as a highly-versatile but auxiliary methodology --- both techniques been described along the lines of "the second best way to do almost anything" \citep{miaoulis2008intelligent, eiben2015introduction}.
However, the utility and ubiquity of ANNs has since increased dramatically.
The development AlexNet is widely considered pivotal to this transformation.
AlexNet united methodological innovations from the field (such as big datasets, dropout, and ReLU) with GPU computing that enabled training of orders-of-magnitude-larger networks.
In fact, some aspects of aspects of their deep learning architecture were expressly modified to accommodate multi-GPU training \citep{krizhevsky2012imagenet}.
By adapting existing methodology to accommodate --- and take advantage of --- commercially available hardware AlexNet spurred the greater availability of compute resources to the research domain and eventually the introduction of custom hardware to expressly support deep learning \citep{jouppi2017datacenter}.

AlexNet set into motion within the field a cycle of repeatedly earning continued investment and developing methodology, software, and eventually hardware to take advantage of additional resources as well as continuing advances of on-silicon design and manufacturing.
In addition to developing hardware-agnostic theory and methodology, we believe that pushing the envelope of open-ended evolution will analogously require expressly designing systems to effectively leverage existing commercially-available parallel and distributed compute resources at circumstantially-feasible scales.
Modern high-performance scientific computing clusters appear perhaps the best target to start down this path.
These systems combine
\begin{itemize}
\item memory-sharing parallel architectures comprising dozens of cores (commonly targeted using OpenMP \citep{dagum1998openmp})
\item and low-latency high-throughput message-passing between distributed nodes (commonly targeted using MPI\citep{clarke1994mpi}).
\end{itemize}

Contemporary scientific computing clusters lack key characteristics highlighted for indefinite scalability: fault tolerance, arbitrary extensibility, and steady asynchronous operation. %TODO check on these last ones.
However, they also may offer an opportunity not available in a indefinitely scalable framework: log-time interconnects \citep{mollah2018comparative}.
\footnote{
The availability of log-time interconnects seems likely to continue in the foreseeable future.
}

\subsection{Instantiating Small-World Networks on Parallel and Distributed Hardware}

Many natural systems --- such as ecosystems, genetic regulatory networks, and neural networks --- are known to exhibit small-world patterns of connectivity or interaction between components \citep{bassett2017small, fox2014herbivores, gaiteri2014beyond}.
In small-world graphs, mean path length (the number of edges traversed on a shortest-route) between arbitrary components scales logarithmically with system size \citep{watts1998collective}.
We anticipate that open-ended phenomena emerging across distributed hardware might also involve small-world connectivity dynamics.
With respect to the latency and bandwidth of inter-component interactions as such a system scales?
What would the impact be of providing a system of hierarchical log-time hardware interconnects as opposed to relying solely on local hardware interconnects?

In Section \ref{sec:results}, we analyze the scaling relationship between system size and expected node-to-node hops traversed between computational elements interacting as part of an emergent small-world network
\footnote{
Although relativistic concerns do ultimately limit latency between spatially-distributed computational elements, with respect to current hardware co-located at a single physical site at foreseeable scales we see minimizing node-to-node hops as key.
Additionally, although we focus on asymptotic analyses, in some cases better scaling coefficients might be achieved with long-distance hardware interconnects.
Maximizing performance in this regard is also an important goal.

In addition, the asymptotic analyses performed on mesh networks without long-distance hierarchical interconnects can also be interpreted in terms of Euclidean distance with respect to relativistic limitations.
}
\begin{enumerate}
\item with and without hierarchical log-time physical interconnects between computational nodes, and
\item with computational nodes embedded on one-, two-, or three-dimensional computational meshes.
\end{enumerate}

In Section \ref{sec:proof1}, we find that expected hops over edges weighted by edge betweenness centrality scales polynomially in all cases without hierarchical physical interconnects.
With hierarchical physical interconnects, a logarithmic scaling relationship can be achieved.

In Section \ref{sec:proof2} and \ref{sec:proof3} we find that hierarchical physical interconnects yield better best-case mean hops per edge in the one-dimensional case.
Interestingly, asymptotically better outcomes in two- and three- dimensions cannot be guaranteed by hierarchical physical interconnects.
This suggests that --- even at truly vast scales --- emergent inter-component interaction networks could arise with bounded per-hardware-component messaging load.

In Section \ref{sec:proof4} we show that, with a specific traditional construction of small-world graphs, best-case mean hops per edge scales polynomially with graph size.
With hierarchical physical interconnects, a logarithmic scaling relationship can be achieved.

These theoretical analyses suggest that whether log-time physical interconnects deliver asymptotically better mean connection latency and hop-efficiency depend on the structure of the network overlaid on a spatially-distributed hardware system.
In either case, log-time physical interconnects may nonetheless provide valuable performance gains. %TODO do I need to expand on/justify this?

\subsection{Exploiting Log-Time Physical Interconnects: a Case Study}

What could an artifical life system that exploits log-time hardware interconnects look like?

We present an extension to upcoming work incorporating genetic programming with the DISHTINY platform for studying evolutionary transitions in individuality \citep{moreno2019toward}.
In existing work with the system, cells situated on a two-dimensional grid interact exclusively with immediate neighbors.
This extension introduces explicitly-registered direct interconnects between (potentially distant) cells for messaging and resource-sharing.
Cells may establish these interconnects through an genetically-mediated exploratory growth process.

In Section \ref{sec:casestudy}, we report a case study where adaptive resource-sharing and messaging emerged over these interconnects.

In future implementations, explicitly-registered cell-cell interconnects may  use log-time physical interconnects.
Our prototype implementation exploits shared-memory thread-level parallelism on a single multiprocessor.

\subsection{Abstraction, Engineering, and the Path of Computational Scale}

Although designed with an eye to scalability largely along the lines outlined by Ackley, DISHTINY exchanges a uniform, evolutionary-passive substrate for manually-engineered self-replicators.
Evolutionary transitions in individuality provide a framework to unite self-replicators and induce meaningful functional synthesis of programmatic components tied to individual compute elements.
This approach mirrors the philosophy of practicality and feasibility laid out by Channon in \citep{channon2019maximum}.

\begin{displayquote}
It is not computationally feasible (even if we knew how) for an OEE simulation to start from a sparse fog of hydrogen and helium and transition to a biological-level era, so it is clearly necessary to skip over or engineer in at least some complex features that arose through major transitions in our universe.
\end{displayquote}

Here, we engineer-over complex features (e.g., genetic transmission of variation, explicitly-registered cell-cell interconnects), aiming to reflect hardware capabilities (e.g., procedural expression of programs, log-time physical interconnects) so they can be fully taken advantage of.

More granular, less prescriptional approaches very well may become ascendant when orders of magnitude of more compute power --- toward the extent envisioned by Ackley --- become available.
Such systems will address important questions in their own right about the computational foundations of physical and biological reality.
Current work developing those systems sets the stage for that eventuality \citep{ackley2018alife}.
