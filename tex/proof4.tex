\subsection{Wiring a Watts-Strogatz Graph} \label{sec:proof4}

Suppose we have a small-world graph $G$ constructed over our usual computational mesh $N$ (as in Figure \ref{fig:nandg}) using the Watts–Strogatz algorithm.
In this procedure, vertices in $V(G)$ corresponding to neighboring computational nodes in $V(N)$ are wired together to form a lattice with mean degree $k$.
Then, for every vertex $v \in V(G)$, each edge $\{x, y\} \in E(G)$ containing $v$ is reconfigured with probability $0 < \beta < 1$ to connect $v$ to a randomly-chosen node $w \in V(G)$.

Before reconfiguration, the total wiring cost of $G$ with respect to hops over $N$ was proportional to $|V(G)|$.

Recall that, with mesh dimensionality $r$ we know that for a pair of arbitrary nodes $a,b \in V(N)$,
\begin{align*}
\bar{d}(a, b) \propto |V(N)|^{\frac{1}{r}} \times r.
\end{align*}

So, after rewiring, the total wiring cost $w$ of $G$ with respect to hops over $N$ can be calculated as
\begin{align*}
  \beta |V(G)| \times |V(N)|^{\frac{1}{r}} \times r + (1 - \beta) |V(G)|.
  %\\
  % \beta |V(N)| \times |V(N)|^{\frac{1}{r}} \times r + (1 - \beta) |V(N)|.
  % \beta |V(N)|^{\frac{r+1}{r}} \times r + (1 - \beta) |V(N)|.
\end{align*}

So, $w \in \Omega \Big( |V(N)|^{\frac{r+1}{r}} \times r \Big).$

In this graph, the number of edges is proportional to the graph size $n$.

With bounded mean degree, we have $|E(G)| \propto |V(G)|$ so we can establish the following lower bound on mean wiring cost per edge of $G$ with respect to hops over $N$,
\begin{align*}
\Omega \Big( |V(G)|^{\frac{1}{r}}| \times r \Big).
\end{align*}

Note that, with the introduction of log-time hierarchical hardware interconnects into $N$ the mean wiring cost per edge of $G$ with respect to hops over $N$ is bounded in the worst case by $\Omega \Big( \log |V(G)| \Big)$.
